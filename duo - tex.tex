%% LyX 2.2.3 created this file.  For more info, see http://www.lyx.org/.
%% Do not edit unless you really know what you are doing.
\documentclass[11pt,twoside,english]{article}
\usepackage[T1]{fontenc}
\usepackage[latin9]{inputenc}
\usepackage[a4paper]{geometry}
\geometry{verbose,tmargin=1in,bmargin=1in,lmargin=1.25in,rmargin=1.25in,headheight=1in,headsep=0.3in,footskip=0.8in}
\setcounter{tocdepth}{2}
\setlength{\parskip}{\bigskipamount}
\setlength{\parindent}{0pt}
\usepackage{babel}
\usepackage{float}
\usepackage{amsmath}
\usepackage{amsthm}
\usepackage{amssymb}
\usepackage{graphicx}
\usepackage[unicode=true]
 {hyperref}

\makeatletter

%%%%%%%%%%%%%%%%%%%%%%%%%%%%%% LyX specific LaTeX commands.
%% Special footnote code from the package 'stblftnt.sty'
%% Author: Robin Fairbairns -- Last revised Dec 13 1996
\let\SF@@footnote\footnote
\def\footnote{\ifx\protect\@typeset@protect
    \expandafter\SF@@footnote
  \else
    \expandafter\SF@gobble@opt
  \fi
}
\expandafter\def\csname SF@gobble@opt \endcsname{\@ifnextchar[%]
  \SF@gobble@twobracket
  \@gobble
}
\edef\SF@gobble@opt{\noexpand\protect
  \expandafter\noexpand\csname SF@gobble@opt \endcsname}
\def\SF@gobble@twobracket[#1]#2{}
%% A simple dot to overcome graphicx limitations
\newcommand{\lyxdot}{.}


%%%%%%%%%%%%%%%%%%%%%%%%%%%%%% Textclass specific LaTeX commands.
\theoremstyle{plain}
\newtheorem{thm}{\protect\theoremname}[section]

\makeatother

\providecommand{\theoremname}{Theorem}

\begin{document}

\title{A Stable Token Based on Dual-Class Structure}

\author{Yizhou CAO\thanks{\textit{Email}: \protect\href{mailto:yizhou.cao@finbook.co}{yizhou.cao@finbook.co},
CTO and Co-founder of FinBook}$\:,$ Min DAI\thanks{\textit{Email}: \protect\href{mailto:matdm@nus.edu.sg}{matdm@nus.edu.sg}.
Director, Centre for Quantitative Finance, National University of
Singapore}$\:,$ Steven KOU\thanks{\textit{Email}: \protect\href{mailto:matsteve@nus.edu.sg}{matsteve@nus.edu.sg}.
Director, Risk Management Institute, National University of Singapore}$\:$, Lewei LI\thanks{\textit{Email}: \protect\href{mailto:lewei.li@finbook.co}{lewei.li@finbook.co},
CEO and Co-founder of FinBook}$\:$, Chen YANG\thanks{\textit{Email}: \protect\href{mailto:chen.yang@math.ethz.ch}{chen.yang@math.ethz.ch},
Department of Mathematics, ETH Zurich}}

\date{Version 0.7.0\\
9 Mar 2018}
\maketitle
\begin{abstract}
Major crypto-currencies are subject to massive volatility, which makes
them attractive to speculators but unsuitable for mainstream use.
There have been several attempts on implementing 'stablecoins'. But
most of them fail to provide steady returns that's appealing to regular
investors.

Inspired by the dual purpose fund that has gained popularity in the
US and China, we propose, for the first time to our best knowledge,
a dual-class crypto token structure that offers entitlements to either
fixed income or leveraged capital appreciation: the income token receives
periodical coupon payments with principal protection; the leverage
token provides leveraged participation in underlying crypto assets.
The whole mechanism is governed by smart contracts run on Ethereum
blockchain, without dealing with centralized counterparties. In theory,
we justify the income token as an ideal stable token.

For illustration, we employ Ether (ETH) as underlying asset and use
its historical prices to demonstrate  the performance of our stable
token in real world scenarios. We also show how arbitrage activities
take place and help maintain price stability of this stable token.
\end{abstract}
\textit{Keywords: }stablecoin, dual-class token, fixed income crypto
asset, leveraged return crypto asset, smart contract, Ethereum, decentralized
asset management

\clearpage{}

\section{Introduction}
\begin{flushright}
{\small{}A wise man ought always to follow the paths beaten by great
men, }\\
{\small{}and to imitate those who have been supreme, so that }\\
{\small{}if his ability does not equal theirs, at least it will savour
of it.}\\
{\small{}\textendash{} }\emph{\small{}Niccolo Machiavelli: The Prince}
\par\end{flushright}{\small \par}

\subsection{Stablecoins and Fixed Income Assets}

Since their inception, crypto currencies aim to create an alternative
financial system comparable to fiat financial systems. In a fiat system,
money or credit exists in a few different forms: actual currency in
circulation and fixed income assets such as time deposits and money
market funds, which are the foundation for rest of the financial market
to build on. The crypto market is at primitive stage as it lacks coins
or tokens comparable to fiat money due to their huge price volatility
and thus there are no established fixed income assets either. 

A Stable token, also known as stablecoin, is a crypto-token that keeps
stable market value against a specific index, most noticeably US Dollar.
There have been a few attempts to create stable tokens: Tether claims
to have 1:1 USD collateral in its centralized bank accounts\cite{USDT};
Basecoin tries to control money supply by playing an algorithmic version
of central bank\cite{Basecoin}; MakerDAO uses decentralized over-collateralization
to maintain confidence of its Dai stablecoin\cite{MAKERDAO}. 

Most stable token designs to date face a common problem: their holders
bear variable levels of counter-party risks, but there is little incentive
or even additional cost to hold these tokens. We believe a crypto-token
with low price volatility and stable income is a superior safe haven
asset and is essential for a more robust crypto market.

\subsection{Risk Segmentation}

In the fiat market, risk segmentation exists in every level of market.
At asset class level, risk averse investors choose fixed income products
while risk seeking investors choose equity or commodity. Within the
fixed income market, conservative investors choose safe haven assets
such as government bonds while aggressive players choose high yield
bonds, or otherwise known as junk bonds. 

Although most of crypto assets are considered highly speculative,
there still exist different segments of risk appetites. The existence
of margin trading in exchanges like BitFinex and Huobi Pro demonstrates
the need of leverage from aggressive crypto investors. Similarly,
safe haven assets providing low volatility and stable income will
be appealing to institutional players, short-term pessimistic investors
and ICO project teams. However, to our best knowledge, there is no
viable product at the moment. 

The price volatility cannot be eliminated, but can be transferred.
Through a structured contract, a risk-seeking investor can get exposed
to higher leverage with limited assets, while a risk-averse investor
can avoid unwanted price fluctuations by entering the other side of
the contract. The financial market has created many products satisfying
different risk segments, some of which can be very practical for crypto
market participants.

\subsection{Leveraged Funds}

Leverage is the strategy of using borrowed capital to increase the
potential return of an investment\cite{Leverage}. Retail investors
have less access to leverage methods such as margin loans and financial
derivatives, which promotes the creation of leveraged fund products. 

An interesting class of leveraged funds is the dual-purpose funds,
also known as primes and scores, which are mutual funds that are split
into two classes of shares by design: low risk shares and high risk
shares. They were designed to provide investors with the opportunity
of repackaging their investment incomes and capital flow in accordance
with their preferences. Dual-purpose funds first appeared in U.S.
in 1965, but faded out in early 1990s due to a change in the tax code
that resulted in double taxation.\cite{DM2018}

The dual-purpose fund was introduced in China in 2010 and it quickly
became one of the most popular leveraged products there, reaching
market size of 500 billion CNY (80 billion USD) at its peak. A Chinese
dual-purpose fund usually takes an open-end index-tracking fund as
underlying and has two classes of shares: the low risk A shares behave
like a perpetual bond with periodical coupon payments, and the high
risk B shares are essentially a closed-end fund magnifying the exposure
of the underlying index. A set of upward and downward reset clauses
is imposed to reduce the risk of both shares.\cite{DM2018}

We are inspired by Chinese dual-purpose fund for three reasons: (i)
both the income and leveraged shares answer unfulfilled demands of
cryptocurrency investors; (ii) the market mechanism, pricing formula,
and arbitrage method prove to work in real market; and (iii) the structure
can be efficiently implemented by Smart Contracts.

\subsection{Smart Contract}

Smart Contract was first proposed by Nick Szabo in 1994, who described
it as \textquotedblleft a set of promises, specified in digital form,
including protocols within which the parties perform on these promises\textquotedblright .

The idea has not received commercial adoption due to lack of trustworthy
execution environment, until the appearance of blockchain technology.
Released in 2015, Ethereum implements a nearly Turing-complete language
on its blockchain, a prominent smart contract framework. The Ethereum
white paper describes smart contracts as \textquotedblleft cryptographic
boxes contain value and only unlock it if certain conditions are met\textquotedbl{}.
It also lists financial derivatives as the most common application
of a smart contract, and introduces a rough idea about creating stable-value
currencies through a hedging contract. \cite{ETH}

Up until now, the most popular stablecoin solution has been issuer-backed
assets. And it is almost a market consensus that such issuers are
not always trustworthy. We agree with Ethereum authors' argument that,
a decentralized market of speculators and arbitragers powered by smart
contract, could be an alternative solution.

This paper aims to propose an alternative approach of creating stable
value tokens, providing a different set of trade offs between price
stability, holder incentive and supply flexibility. We believe the
solution will be very useful for broader adoption of the crypto market.

\clearpage{}

\section{Product Design\label{sec:Product-Design}}
\begin{flushright}
{\small{}Everything should be made as simple as possible, but no simpler.
}\\
{\small{}\textendash{} }\emph{\small{}Albert Einstein}
\par\end{flushright}{\small \par}

\subsection{Overview}

\paragraph{DUO }

a dual-class token structure that, combined with smart contract governed
rules and market arbitrage mechanism, provides principal-guaranteed
fixed incomes and leveraged capital gains for holders of each class,
respectively.

\paragraph{Class A Token}

also known as the Income Token, continuously accumulates interests
based on its original net value at last Reset event. It will also
receive token payments at each Reset event.

\paragraph{Class B Token}

also known as the Leverage Token, entitles leveraged participation
of the underlying digital assets.

The Custodian smart contract performs multiple tasks that facilitate
key mechanism of the system, including: creation and redemption of
DUO tokens, safekeeping the underlying digital assets (e.g. ETH),
calculation of tokens' net values, and execution of Reset events. 

\subsection{Creation and Redemption\label{subsec:Creation-and-Redemption}}

\subsubsection{Creation}

Dual-class tokens can be created by depositing underlying tokens to
the Custodian contract. Upon receiving underlying tokens of amount
$M_{C}$, the Custodian contract will return to the sender equal amount
of Class A and Class B tokens. Such amount $C$ can be calculated
by:
\begin{equation}
\begin{array}{c}
C=\frac{M_{C}\cdot P_{0}}{2}\ ,\end{array}\label{eq:creation}
\end{equation}
where $P_{0}$ is the recorded price of underlying token in USD at
last reset event.

The deposited underlying tokens are kept by the Custodian contract,
as collateral of the Class A and Class B tokens issued by the contract.
Any user or member of the public can verify the collateral and tokens
issued through third party applications such as \href{http://etherscan.io}{Etherscan.io}.

\subsubsection{Redemption}

Holders of Class A and Class B tokens can withdraw deposited underlying
tokens at any time by performing a redemption. To do this, the user
will send equal amount $R$ of Class A and Class B tokens to the Custodian
contract. The contract will deduct Class A and Class B tokens, and
return to the sender $M_{R}$ underlying tokens, where$M_{R}$ can
be calculated by:
\begin{equation}
M_{R}=\frac{2R}{P_{0}}\ .\label{eq:redemption}
\end{equation}

\subsubsection{Net Value}

The net values of tokens are calculated based on the coupon rate,
the elapsed time from last Reset event, and the latest underlying
token price in USD fed to the system. In particular,
\begin{equation}
\begin{array}{cc}
V_{A}^{t}= & 1+r\cdot t\\
V_{B}^{t}= & 2\cdot\frac{P_{t}}{P_{0}}-V_{A}^{t}\ ,
\end{array}\label{eq:netvalue}
\end{equation}
where $r$ is the \textbf{daily} coupon rate,

$t$ is the number of days from last reset event,

and $P_{t}$ is the current price of underlying token in USD.

\subsubsection{Implied Leverage Ratio}

\begin{align*}
L_{B}^{t} & =\frac{P_{t}}{P_{0}}\cdot\frac{2}{V_{B}^{t}}
\end{align*}
Note that at inception or after resets, above simply reduces to $L_{B}^{0}=2$.

\subsection{Resets}

Simply put, the holders of Class B tokens borrow capital from the
holders of Class A tokens and invest in a volatile asset (i.e. ETH),
which leads the Class B token to possess a continuum of leverage ratios.
To reduce risk of both classes, a set of upward and downward reset
clauses is imposed. Resets preserves total value in the system.

\subsubsection{Contingent Upward Reset}

When underlying token price rises, the Class B token's leverage ratio
reduces. An upward reset is triggered when the Net Value of Class
B token reaches upper limit $\mathcal{H}_{u}$. The reset will restore
the leverage ratio to 2. By increasing Class B's leverage ratio, the
structure retains attraction to leverage-seeking users.

Upon upward reset:
\begin{enumerate}
\item Total amount of both classes token remain unchanged
\item Net Value of both classes reset to 1 USD
\item Both classes' holders will receive certain amount of underlying token
from the Custodian contract. Such amount for each Class A token is
$\frac{V_{A}^{t-}-1}{P_{t}}$ and for each Class B token is $\frac{V_{B}^{t-}-1}{P_{t}}$
\end{enumerate}
Total value in the system is unchanged after reset. For details, see
Appendix \ref{subsec:Contingent-Upward-Reset}.

\subsubsection{Contingent Downward Reset}

When the underlying token price falls, the Class B token's leverage
ratio increases. A Downward Reset is triggered when the Net Value
of Class B token reaches lower limit $\mathcal{H}_{d}$. Similar to
an Upward Reset, a Downward Reset restores the leverage ratio. By
reducing leverage ratio and restoring Net Value, the structure strengthens
principal protection to Class A holders. Additionally, a Downward
Reset will reduce the total supply of Class A and Class B tokens by
roughly $1-\mathcal{H}_{d}$. 

Upon downward reset:
\begin{enumerate}
\item Total amount of both classes token are reduced to $Q^{t+}=Q^{t-}\cdot V_{B}^{t-}$
\item Net Value of both classes reset to 1 USD
\item Class A holders will receive certain amount of underlying token from
the Custodian contract. Such amount of each Class A token is: $\frac{V_{A}^{t-}-V_{B}^{t-}}{P_{t}}$
\end{enumerate}
Total value in the system is unchanged after reset. For details, see
Appendix  \ref{subsec:Contingent-Downward-Reset}.

\subsubsection{Periodic Reset}

When a given time period has passed from the last reset time, $V_{A}^{t-}$
grows to upper limit $\mathcal{H}_{p}$, a periodic reset is triggered
to restore the leverage ratio and net value of both classes. If $V_{B}^{t-}$
is less than 1, reset will be the same as the Contingent Downward
reset. Otherwise, reset will be the same as the Contingent Upward
reset.

\clearpage{}

\section{Valuation}

Denote $W_{A}$ and $W_{B}$ as the market value of Class A and Class
B, respectively. We assume that $P$ follows a geometric Brownian
motion under the risk neutral measure: 
\[
dP_{t}=\rho Pdt+\sigma P_{t}dW_{t},
\]
where $W_{t}$ is a one-dimensional standard Brownian motion. Also
denote$S_{t}=P_{t}/P_{0}$, then 
\[
dS_{t}=\rho S_{t}dt+\sigma S_{t}dW_{t},
\]
where $\rho$ denotes the risk-free rate. Note that $S_{\theta}=1$
at any reset time $\theta$, when $P_{0}$ is reset to $P_{\theta}$.

Based on the above specifications, the current market value of A shares
under the risk-neutral pricing framework is given recursively as 
\begin{align}
\begin{split}W_{A}(t,S) & =E_{t}^{*}\Bigg[e^{-\rho(\zeta-t)}[rT+g(S_{\zeta-})+(1-g(S_{\zeta-}))W_{A}(0,1)]\cdot\mathbf{1}_{\{\zeta\le\tau,\eta\}}\\
 & \quad+e^{-\rho(\tau-t)}(r\tau+W_{A}(0,1))\cdot\mathbf{1}_{\{\tau<\eta,\zeta\}}\\
 & \quad+e^{-\rho(\eta-t)}(r\eta+1-\mathcal{H}_{d}+\mathcal{H}_{d}W_{A}(0,1))\cdot\mathbf{1}_{\{\eta<\tau,\zeta\}}\Bigg],
\end{split}
\label{VAL2}
\end{align}
where $E_{t}^{*}$ is the $\mathbb{Q}$-expectation computed under
the initial condition $S_{t-}=S$, random times $\zeta$, $\tau$
and $\eta$ represent the first periodic, upward and downward reset
date from $t$, respectively, and $g(s)=((1+\alpha)(1-s)+\alpha rT)^{+}$.
Here $W_{A}(t,S)$ denotes the market value of A share with time from
last interest payment $0\le t\le1$, and underlying fund value $H_{d}(t)\le S\le H_{u}(t)$,
where $H_{d}(t)=\frac{\alpha}{1+\alpha}(1+rt)+\frac{1}{1+\alpha}\mathcal{H}_{d}$,
and $H_{u}(t)=\frac{\alpha}{1+\alpha}(1+rt)+\frac{1}{1+\alpha}\mathcal{H}_{u}$
Note that by labeling the last interest payment date as time 0, $t$
can also be interpreted as the current time.

Since after either an upward or downward reset, investors still get
Class A shares in addition to the payments, the right hand side of
$\eqref{VAL2}$ depends on $W_{A}$ which makes $\eqref{VAL2}$ recursive.
On the right hand side of ($\ref{VAL2}$), the second and third terms
describe the cash flow on upward and downward resets. The second term
shows that, if an upward reset comes first, investors get early payment,
and the time from last reset changes to 0 and $S$ to $1$. The third
term shows that, if a downward reset comes first, investors receive
payment of value $V_{A}^{\eta-}-V_{B}^{\eta-}$, each share of Class
A reduces to $\mathcal{H}_{d}$ share, and the time from last reset
changes to 0 and $S$ to $1$. Finally, the first term is value of
the cash flow on the periodic reset date. Specifically, when $V_{B}^{\zeta-}\ge1$,
or equivalently, $(1+\alpha)-\alpha(1+rT)\ge1$ (so that $g(S_{\zeta-})=0$),
the cash flow follows that of an upward reset where Class A receives
payment of value $rT$, and the number of Class A remains unchanged;
when $V_{B}^{\zeta-}<1$, or equivalently, $(1+\alpha)-\alpha(1+rT)<1$,
the cash flow follows that of a downward reset, where Class A receives
payment of value $V_{A}^{\zeta-}-V_{B}^{\zeta-}=1+rT-(1+\alpha)S_{\zeta-}-\alpha(1+rT)=rT+g(S_{\zeta-})$,
and each one share of Class A is reduced to $V_{B}^{\zeta-}=(1+\alpha)S_{\zeta-}-\alpha(1+rT)=1-g(S_{\zeta-})$
share.

The following theorem characterizes $W_{A}$ in terms of a partial
differential equation.
\begin{thm}
$W_{A}$ is the unique classical solution\footnote{By classical solution we mean $W_{A}\in C^{1,2}(Q)\cap C(\overline{Q}\backslash D)$,
where $Q=\{(t,S):0\le t<1,H_{d}(t)<S<H_{u}(t)\}$ and $D=\{T\}\times\{H_{d}(T),H_{u}(T)\}$.} to the following partial differential equation on $\{(t,S):0\le t<T,H_{d}(t)<S<H_{u}(t)\}$
\begin{align}\label{eqn:PDEstart} -\frac{\partial W_{A}}{\partial t} & =\frac{1}{2}\sigma^{2}S^{2}\frac{\partial^{2}W_{A}}{\partial S^{2}}+\rho S\frac{\partial W_{A}}{\partial S}-\rho W_{A}\\ \label{eqn:ter} W_{A}(T,S) & =rT+g(S)+(1-g(S))W_{A}(0,1)\\ \label{eqn:bound1} W_{A}(t,H_{u}(t)) & =rt+W_{A}(0,1)\\\label{eqn:PDEend} W_{A}(t,H_d(t)) & =rt+1-\mathcal{H}_{d}+\mathcal{H}_{d}W_{A}(0,1). \end{align} 
\end{thm}
The proof follows that of Theorem 4.1 in \cite{DM2018} by adopting
the change in the terminal condition. The main feature of the above
PDE problem is the nonlocal terminal and boundary conditions $\eqref{eqn:ter}$
\textendash{} $\eqref{eqn:PDEend}$, where the given data also depend
on the solution $W_{A}$ itself. Although $\eqref{eqn:PDEstart}$
involves the standard Black-Scholes operator (due to our geometric
Brownian motion assumption), the presence of the solution $W_{A}$
in terminal and boundary data makes the PDE significantly different
from the classical Black-Scholes PDE, leading to challenges in both
theoretical and numerical aspects. On the theoretical aspect, the
existing stochastic representation result is not applicable to PDE
$\eqref{eqn:ter}$ \textendash{} $\eqref{eqn:PDEend}$ due to the
nonlocalness of the terminal and boundary conditions. On the numerical
aspect, the nonlocalness in the terminal and boundary data also causes
problem since a numerical solution in the interior region depends
on the boundary and terminal data, which in turns depends on the numerical
solution in the interior region. To solve this, we propose an iterative
procedure presented in Appendix $\ref{subsec:Numerical-Procedure-for}$. 

The nonlocal terminal and boundary conditions in $\eqref{eqn:ter}$
\textendash{} $\eqref{eqn:PDEend}$ are directly related with the
cash flow of A shares. The upper boundary condition ($\ref{eqn:bound1}$)
at $S=H_{u}(t)$ corresponds to the upward reset, when early payment
$Rt$ is delivered and $S$ resets to 1; the lower boundary condition
($\ref{eqn:PDEend}$) at $S=H(t)$ corresponds downward reset, when
early payment $V_{B}-V_{A}$ is delivered, each one A share shrinks
to $H_{d}$ share, and $S$ resets to 1. Finally, the terminal condition
($\ref{eqn:ter}$) corresponds to the periodic reset, where the payment
follows either the upward or downward reset based on whether $V_{B}\le1$.

\clearpage{}

\section{ETH Examples}

For illustration, we hereby uses Ether (ETH) as the underlying token
and apply below parameter values:
\begin{align*}
r\;\; & =0.02\%\\
\mathcal{H}_{d} & =0.25\\
\mathcal{H}_{u} & =2\\
\mathcal{H}_{p} & =1.02\ ,
\end{align*}
and below assumptions are used:
\begin{enumerate}
\item Price is monitored on \textbf{daily} basis
\item Resets are performed according to end-of-day prices
\item The structure starts with 100,000 ETH with no further capital injection
\item Reinvestment of ETH payout from resets is not considered
\end{enumerate}

\subsection{Overall Market}

ETH/USD price data from 1 Oct 2017 to 28 Feb 2018 is used. In this
period, ETH started from 303.95 USD to low of 284.92 USD and high
of 1,359.48 and ended at 851.5. 

At inception, 15,197,500 Class A and 15,197,500 Class B are issued
from the 100,000 ETH, total value in the system is 30,395,000 USD.

Three upward resets were triggered on 24 Nov 2017 ($V_{B}=2.0846$),
17 Dec 2017 ($V_{B}=2.0467$) and 7 Jan 2018 ($V_{B}=2.1106$) while
one downward reset was triggered on 5 Feb 2018 ($V_{B}=0.2379$.).
No periodic reset was triggered in the sampling period.\\

\begin{figure}[H]
\begin{centering}
\includegraphics[scale=0.9]{2\lyxdot 4_TotalValue}
\par\end{centering}
\caption{Total Values}

\end{figure}

\begin{figure}[H]
\begin{centering}
\includegraphics[scale=0.9]{2\lyxdot 4_NetValue}
\par\end{centering}
\caption{Net Values}
\end{figure}

\paragraph{Period without Reset}

The total value of Class A and Class B tokens moves in line with ETH
price. Class B tokens get exposed to full price fluctuations of ETH,
while Class A token value is stable and slowly appreciating.\\

\begin{figure}[H]
\begin{centering}
\includegraphics[scale=0.9]{2\lyxdot 4_No_Reset_TotalValue}
\par\end{centering}
\caption{1 Oct 2017 - 23 Nov 2017 Total Values}
\end{figure}

\paragraph{Period with Contingent Upward Reset\protect\footnote{Chart and numbers are similar for the other two upward resets and
thus omitted here.}}

17 Dec 2017, $V_{B}=2.0467$. 97.4049 ETH is paid out to Class A holders
and 22,163.7213 ETH is paid out to Class B holders. Remaining System
value is at 30,395,000 USD. The total value of Class A, Class B and
ETH payout is consistent with ETH price movement.\\

\begin{figure}[H]
\begin{centering}
\includegraphics[scale=0.9]{2\lyxdot 4_Upward_TotalValue}
\par\end{centering}
\caption{24 Nov 2017 - 6 Jan 2017 Total Values}
\end{figure}

\paragraph{Period with Contingent Downward Reset}

5 Feb 2018, $V_{B}=0.2379$. 16,789.3847 ETH is paid out to Class
A holders. total amount of both classes is reduced to 3,615,680.02.
Remaining System value is at 7,231,360.04 USD. The total value of
Class A, Class B and ETH payout is consistent with ETH price movement.\\

\begin{figure}[H]
\begin{centering}
\includegraphics[scale=0.9]{2\lyxdot 4_Downward_TotalValue}
\par\end{centering}
\caption{7 Jan Nov 2017 - 28 Feb 2017 Total Values}
\end{figure}

\subsection{Individual Class Return}

\paragraph{Class A}

stable return from daily coupon and not affected by ETH price movement.\\

\begin{figure}[H]
\begin{centering}
\includegraphics[scale=0.9]{2\lyxdot 4_A_Return}
\par\end{centering}
\caption{Class A Return}
\end{figure}

\paragraph{Class B}

magnified return from ETH price movement due to the embedded leverage.\\

\begin{figure}[H]
\begin{centering}
\includegraphics[scale=0.9]{2\lyxdot 4_B_Return}
\par\end{centering}
\caption{Class B Return}
\end{figure}

\paragraph{Note}

In order to achieve above return, ETH payout from resets need to be
fully converted back to the corresponding class token. For example,
after 24 Nov 2017 upward reset, all ETH received by Class A holder
need to be converted into Class A token.

\subsection{Market Value}

\begin{figure}[H]
\begin{centering}
\includegraphics[scale=0.6]{WA-eps-converted-to}
\par\end{centering}
\caption{Class A Market Value}
\end{figure}

\begin{figure}[H]
\begin{centering}
\includegraphics[scale=0.6]{WB-eps-converted-to}
\par\end{centering}
\caption{Class B Market Value}
\end{figure}

\clearpage{}

\section{Market Mechanism}
\begin{flushright}
{\small{}Arbitrage human nature. It's not going to change any time
soon. }\\
{\small{}\textendash{} }\emph{\small{}James O'Shaughnessy}
\par\end{flushright}{\small \par}

\subsection{Premium and Discount\label{subsec:Premium-and-Discount}}

Like all fixed income products, Class A is expected to trade on premium
or at discount with respect to its net value, depending on market
condition. When Class A trades on premium, that generally indicates
a higher demand for income over leverage and thus Class A buyers are
getting lower yield. On the contrary, when Class A trades at discount,
that implies a higher demand for leverage over income and thus Class
A buyers are paying less and getting higher yield. In an orderly market,
Class A's market price can be used to derive the current market cost
of funding. As net values draw closer to resets, Class A might trade
closer to its net value as traders start exploiting arbitrage opportunities.

\subsection{Market Arbitrage }

The dual class tokens are two-way fungible to the underlying token
(i.e. ETH). Thus, a price parity shall hold as below:
\[
P_{A}^{t}+P_{B}^{t}=V_{A}^{t}+V_{B}^{t}\ ,
\]
where $P_{A}^{t}$ is the current price of Class A in USD,

and $P_{B}^{t}$ is the current price of Class B in USD.

Along with Equation (\ref{eq:netvalue}) in Section \ref{subsec:Creation-and-Redemption},
below should hold:
\[
P_{A}^{t}+P_{B}^{t}=2\cdot\frac{P_{t}}{P_{0}}\ .
\]
On open market, arbitrage opportunity exists when Class A and Class
B tokens trade away from their net values and the above equation is
likely violated on either side.

\paragraph{Arbitrage via Creation}

When combined market price of Class A and Class B tokens are higher
than their combined net values, i.e.
\[
P_{A}^{t}+P_{B}^{t}>2\cdot\frac{P_{t}}{P_{0}}\ ,
\]
arbitrage profit can be exploited by perform DUO token creation:
\begin{enumerate}
\item Buy 1 underlying tokens from open market for $P_{t}$ USD equivalent.
\item Send the underlying tokens to the Custodian contract.
\item Based on Equation (\ref{eq:creation}), receive $\frac{P_{0}}{2}$
Class A and $\frac{P_{0}}{2}$ Class B tokens.
\item Sell Class A and Class B tokens on open market for $P_{0}\cdot\frac{P_{A}^{t}+P_{B}^{t}}{2}$
USD equivalent.
\end{enumerate}
The expected payoff of above operations is:
\[
P_{0}\cdot\frac{P_{A}^{t}+P_{B}^{t}}{2}-P_{t}=\frac{P_{0}}{2}\cdot\left(P_{A}^{t}+P_{B}^{t}-2\cdot\frac{P_{t}}{P_{0}}\right)>0\ .
\]

\paragraph{Arbitrage via Redemption}

When combined market price of Class A and Class B tokens are lower
than their combined net values, i.e.
\[
P_{A}^{t}+P_{B}^{t}<2\cdot\frac{P_{t}}{P_{0}}\ ,
\]
arbitrage profit can be exploited by perform DUO token redemption:
\begin{enumerate}
\item Buy $\frac{P_{0}}{2}$ Class A and $\frac{P_{0}}{2}$ Class B tokens
from open market for $P_{0}\cdot\frac{P_{A}^{t}+P_{B}^{t}}{2}$ USD
equivalent.
\item Send Class A and Class B tokens to the Custodian contract.
\item Based on Equation (\ref{eq:redemption}), receive 1 underlying tokens.
\item Sell underlying tokens on open market for $P_{t}$ USD equivalent.
\end{enumerate}
The expected payoff of above operations is:
\[
P_{t}-P_{0}\cdot\frac{P_{A}^{t}+P_{B}^{t}}{2}=\frac{P_{0}}{2}\cdot\left(2\cdot\frac{P_{t}}{P_{0}}-\left(P_{A}^{t}+P_{B}^{t}\right)\right)>0\ .
\]
Many other factors also determine whether above arbitrage strategy
can be successfully executed, such as bid ask spread of all the 3
components, cost in creation and redemption, and ability to short
sell the required components. Please refer to appendix for discussion
of a more general case.

\clearpage{}

\section{Further Development}
\begin{flushright}
{\small{}Patience, persistence and perspiration make an unbeatable
combination for success.}\\
{\small{}\textendash{} }\emph{\small{}Napoleon Hill}
\par\end{flushright}{\small \par}

This paper has outlined the main design and market mechanism for the
dual-class token structure. Further studies can be done in below aspects.

\subsubsection*{Underlying Price Pair}

While the dual class structure is independent of the underlying crypto
fiat price pair, the liquidity and popularity of the underlying price
pair do impact the viability of the structure as market arbitrage
is important to ensure the structure trades as designed. In this paper,
ETH/USD is used as the underlying price pair, but other popular ERC20
tokens, such as EOS, ADA, paired with major fiat other than USD, can
also be considered.

\subsubsection*{Base Rate Discovery}

Arbitrary rate can be used for Class A coupon and it will be traded
on premium or at discount based on the market required rate of return.
However, it is desirable that Class A trades close to its net value,
which means the coupon rate for Class A should be chosen close to
the market rate. Currently there are few observable proxies in the
market. Several centralized exchanges allowing margin trading are
charging USD borrow rates in the range of 20\% to 40\% per annum. 

As discussed in Section \ref{subsec:Premium-and-Discount}, the market
premium or discount of Class A token may imply USD's borrow rate in
this market. As the structure gains attractions, its implied rate
could serve as an indication for other USD borrow practice in crypto
market.

\subsubsection*{Optimal Reset Thresholds}

It is important to keep leverage for Class B within certain range:
not too high so as to protect Class A holders from sudden price drop
and not too low so as to keep Class B attractive to leveraged users.
However, it is undesirable that resets, especially downward resets,
happen too frequently.

There are two main parameters to be determined, the ratio between
Class A and Class B shares, $\alpha$ (see Appendix \ref{subsec:General-Product-Design});
and the lower limit of Class B net value, $\mathcal{H}_{d}$ 
, that triggers downward resets. The Chinese market over the years
has concluded a broadly accepted set of parameters: $\alpha$ is set
to 1, meaning the quantities of Class A versus Class B is 1:1; $\mathcal{H}_{d}$
being 0.25, meaning a downward reset will be triggered if the underlying
price dropped approximately 37.5\% from last reset. This setup has
been tested in extreme market events such as the mid-2015 market crash
and the early-2016 circuit breaker turmoil, where the reset clauses
were all successfully implemented and protected the interests of Class
A holders.

We witness that cryptocurrencies have considerably higher volatility
than stock market indices. A series of back testings and Monte Carlo
simulations will be performed to investigate the optimal parameters
for dual class token structures. 

\clearpage{}

\appendix

\section{Appendix}

\subsection{General Product Design\label{subsec:General-Product-Design}}

In Section \ref{sec:Product-Design}, we have described a specific
product design where Class A is stable relative to USD as target fiat
currency and Class B has initial leverage as 2. In addition, transaction
cost in creation and redemption is omitted. In this section, a general
case is discussed.

\subsubsection{Creation}

Dual-class tokens can be created by depositing underlying tokens to
the Custodian contract. Upon receiving underlying tokens of amount
$M_{C}$, the Custodian contract will return to the sender certain
amount of Class A and Class B tokens. Such amount $C_{A}$ and $C_{B}$
can be calculated by:
\begin{equation}
\begin{array}{cc}
C_{B}= & \frac{M_{C}\cdot P_{0}\cdot\left(1-c\right)}{1+\alpha}\\
C_{A}= & C_{B}\cdot\alpha\ ,
\end{array}\label{eq:creation-1}
\end{equation}
where $c$ is the processing fee of the smart contract,

$\alpha$ is a positive number to determine the ratio of A and B,

and $P_{0}$ is the recorded price of underlying token in target fiat
currency at last reset event.

\subsubsection{Redemption}

Holders of Class A and Class B tokens can withdraw deposited underlying
tokens at any time by performing a redemption. To do this, the user
will send amount of $R\cdot\alpha$ Class A and amount of $R$ Class
B tokens to the Custodian contract. The contract will deduct Class
A and Class B tokens, and return to the sender $M_{R}$ underlying
tokens, where$M_{R}$ can be calculated by:
\begin{equation}
M_{R}=\frac{R\cdot\left(1-c\right)\cdot\left(1+\alpha\right)}{P_{0}}\ .\label{eq:redemption-1}
\end{equation}

\subsubsection{Net Value}

The net value of tokens are calculated based on the coupon rate, the
elapsed time from last reset event, and the latest underlying token
price in target fiat currency fed to the system. In particular:
\begin{equation}
\begin{array}{cc}
V_{A}^{t}= & 1+r\cdot t\\
V_{B}^{t}= & \left(1+\alpha\right)\cdot\frac{P_{t}}{P_{0}}-\alpha\cdot V_{A}^{t}\ ,
\end{array}\label{eq:netvalue-1}
\end{equation}
where $r$ is the daily coupon rate,

$t$ is the number of days from last reset event,

and $P_{t}$ is the current price of underlying token in target fiat
currency.

\subsubsection{Quantity}

Below holds in the system at all time
\[
Q_{A}^{t}=Q_{B}^{t}\cdot\alpha\ ,
\]
where $Q_{A}^{t}$ and $Q_{B}^{t}$ are the total amount of Class
A and Class B tokens.

\subsubsection{Implied Leverage Ratio}

\begin{align*}
L_{B}^{t} & =\frac{P_{t}}{P_{0}}\cdot\frac{\left(1+\alpha\right)}{V_{B}^{t}}
\end{align*}
Note that at inception or after resets, above simply reduces to $L_{B}^{0}=1+\alpha$.

\subsubsection{Contingent Upward Reset\label{subsec:Contingent-Upward-Reset}}

An upward reset is triggered when $V_{B}^{t}\geqslant\mathcal{H}_{u}$.
Upon upward reset:
\begin{enumerate}
\item Total amount of both classes token remain unchanged, $Q_{A}^{t+}=Q_{A}^{t-}$
and $Q_{B}^{t+}=Q_{B}^{t-}$.
\item Net Value of both classes reset to 1 target fiat currency.
\item Both classes' holders will receive certain amount of underlying token
from the Custodian contract. Such amount for each Class A token is
$U_{A}=\frac{V_{A}^{t-}-1}{P_{t}}$ and for each Class B token is
$U_{B}=\frac{V_{B}^{t-}-1}{P_{t}}$.
\end{enumerate}
Total value in the system is unchanged after reset:
\begin{align*}
 & U_{A}\cdot P_{t}\cdot Q_{A}^{t-}+U_{B}\cdot P_{t}\cdot Q_{B}^{t-}+Q_{A}^{t+}\cdot V_{A}^{t+}+Q_{B}^{t+}\cdot V_{B}^{t+}\\
= & \left(V_{A}^{t-}-1\right)\cdot Q_{A}^{t-}+\left(V_{B}^{t-}-1\right)\cdot Q_{B}^{t-}+Q_{A}^{t-}\cdot1+Q_{B}^{t-}\cdot1\\
= & V_{A}^{t-}\cdot Q_{A}^{t-}+V_{B}^{t-}\cdot Q_{B}^{t-}\ .
\end{align*}

\subsubsection{Contingent Downward Reset\label{subsec:Contingent-Downward-Reset}}

A downward reset is triggered when $V_{B}^{t}\leqslant\mathcal{H}_{d}$.
Upon downward reset:
\begin{enumerate}
\item Total amount of Class B token is reduced to $Q_{B}^{t+}=Q_{B}^{t-}\cdot V_{B}^{t-}$.
\item Total amount of Class A token is reduced to $Q_{A}^{t+}=Q_{B}^{t+}\cdot\alpha$.
\item Net Value of both classes reset to 1 target fiat currency.
\item Class A holders will receive certain amount of underlying token from
the Custodian contract. Such amount of each Class A token is: $D_{A}=\frac{V_{A}^{t-}-V_{B}^{t-}}{P_{t}}$.
\end{enumerate}
Total value in the system is unchanged after reset:
\begin{align*}
 & D_{A}\cdot P_{t}\cdot Q_{A}^{t-}+Q_{A}^{t+}\cdot V_{A}^{t+}+Q_{B}^{t+}\cdot V_{B}^{t+}\\
= & \left(V_{A}^{t-}-V_{B}^{t-}\right)\cdot Q_{A}^{t-}+Q_{B}^{t+}\cdot\alpha\cdot1+Q_{B}^{t+}\cdot1\\
= & V_{A}^{t-}\cdot Q_{A}^{t-}-V_{B}^{t-}\cdot Q_{B}^{t-}\cdot\alpha+Q_{B}^{t-}\cdot V_{B}^{t-}\cdot\alpha+Q_{B}^{t-}\cdot V_{B}^{t-}\\
= & V_{A}^{t-}\cdot Q_{A}^{t-}+V_{B}^{t-}\cdot Q_{B}^{t-}\ .
\end{align*}
Note above used the fact $Q_{A}^{t-}=Q_{B}^{t-}\cdot\alpha$.

\subsubsection{Periodic Reset}

A periodic reset is triggered when $V_{A}^{t}\geqslant\mathcal{H}_{p}$.
If $V_{B}^{t-}$ is less than 1, reset will be the same as the Contingent
Downward reset. Otherwise, reset will be the same as the Contingent
Upward reset.

\subsection{General Market Arbitrage}

The price parity shall hold as below:
\[
\alpha\cdot P_{A}^{t}+P_{B}^{t}=\alpha\cdot V_{A}^{t}+V_{B}^{t}\ ,
\]
where $P_{A}^{t}$ is the current price of Class A in target fiat
currency,

and $P_{B}^{t}$ is the current price of Class B in target fiat currency.

Along with Equation (\ref{eq:creation-1}), (\ref{eq:redemption-1})
and (\ref{eq:netvalue-1}) in Section \ref{subsec:General-Product-Design},
below should hold:
\[
\frac{P_{t}}{P_{0}}\cdot\left(1-c\right)\leq\frac{\alpha\cdot P_{A}^{t}+P_{B}^{t}}{1+\alpha}\leq\frac{P_{t}}{P_{0}}\cdot\frac{1}{1-c}\ .
\]
On open market, arbitrage opportunity exists when Class A and Class
B tokens trade away from their net values and above equation is violated
on either side.

\paragraph{Arbitrage via Creation}

When combined market price of Class A and Class B tokens are higher
than their combined net values, i.e.
\[
\frac{\alpha\cdot P_{A}^{t}+P_{B}^{t}}{1+\alpha}>\frac{P_{t}}{P_{0}}\cdot\frac{1}{1-c}\ ,
\]
arbitrage profit can be exploited by perform DUO token creation:
\begin{enumerate}
\item Buy $\frac{1}{1-c}$ underlying tokens from open market for $P_{t}\cdot\frac{1}{1-c}$
target fiat currency equivalent.
\item Send the underlying tokens to the Custodian contract.
\item Based on Equation (\ref{eq:creation-1}), receive $P_{0}\cdot\frac{\alpha}{1+\alpha}$
Class A and $P_{0}\cdot\frac{1}{1+\alpha}$ Class B tokens.
\item Sell Class A and Class B tokens on open market for $P_{0}\cdot\frac{\alpha P_{A}^{t}+P_{B}^{t}}{1+\alpha}$
target fiat currency equivalent.
\end{enumerate}
The expected payoff of above operations is:
\[
P_{0}\cdot\frac{\alpha P_{A}^{t}+P_{B}^{t}}{1+\alpha}-\frac{P_{t}}{1-c}=P_{0}\cdot\left(\frac{\alpha\cdot P_{A}^{t}+P_{B}^{t}}{1+\alpha}-\frac{P_{t}}{P_{0}}\cdot\frac{1}{1-c}\right)>0\ .
\]

\paragraph{Arbitrage via Redemption}

When combined market price of Class A and Class B tokens are lower
than their combined net values, i.e.
\[
\frac{P_{t}}{P_{0}}\cdot\left(1-c\right)>\frac{\alpha\cdot P_{A}^{t}+P_{B}^{t}}{1+\alpha}\ ,
\]
similar arbitrage profit can be exploited by perform DUO token redemption:
\begin{enumerate}
\item Buy $P_{0}\cdot\frac{\alpha}{1+\alpha}$ Class A and $P_{0}\cdot\frac{1}{1+\alpha}$
Class B tokens from open market for $P_{0}\cdot\frac{\alpha P_{A}^{t}+P_{B}^{t}}{1+\alpha}$
target fiat currency equivalent.
\item Send Class A and Class B tokens to the Custodian contract.
\item Based on Equation (\ref{eq:redemption-1}), receive $1-c$ underlying
tokens.
\item Sell underlying tokens on open market for $P_{t}\cdot\left(1-c\right)$
target fiat currency equivalent.
\end{enumerate}
The expected payoff of above operations is:
\[
P_{t}\cdot\left(1-c\right)-P_{0}\cdot\frac{\alpha P_{A}^{t}+P_{B}^{t}}{1+\alpha}=P_{0}\cdot\left(\frac{P_{t}}{P_{0}}\cdot\left(1-c\right)-\frac{\alpha\cdot P_{A}^{t}+P_{B}^{t}}{1+\alpha}\right)>0\ .
\]

\subsection{Derivation of the Pricing Equation}

\subsection{Numerical Procedure for the Pricing Equation $\eqref{eqn:PDEstart}$
\textendash{} $\eqref{eqn:PDEend}$\label{subsec:Numerical-Procedure-for}}

We propose an iterative algorithm to obtain a numerical solution of
the periodic parabolic terminal-boundary value problem ($\ref{eqn:PDEstart}$)
\textendash{} ($\ref{eqn:PDEend}$).

\subsection*{Algorithm 1}
\begin{enumerate}
\item \textit{Set the initial guess $W_{A}^{(0)}=0$; }
\item \textit{For $i=1,2,\cdots$: Given $V_{A}^{(i-1)}$, solve for $W_{A}^{(i)}$,
the solution to the equation 
\[
\begin{array}{cccc}
-\frac{\partial W_{A}}{\partial t}= & \frac{1}{2}\sigma^{2}S^{2}\frac{\partial^{2}W_{A}}{\partial S^{2}}+\rho S\frac{\partial W_{A}}{\partial S}-\rho W_{A} &  & 0\le t<T,H_{d}(t)<S<H_{u}(t)\\
W_{A}(1,S)= & rT+g(S)+(1-g(S))W_{A}^{(i-1)}(0,1) &  & H_{d}(t)<S<H_{u}(t)\\
W_{A}(t,H_{u}(t))= & rt+W_{A}^{(i-1)}(0,1) &  & 0\le t\le T\\
W_{A}(t,H_{d}(t))= & rt+1-\mathcal{H}_{d}+\mathcal{H}_{d}W_{A}^{(i-1)}(0,1) &  & 0\le t\le T.
\end{array}
\]
}
\item \textit{If $||W_{A}^{(i)}-W_{A}^{(i-1)}||<$ tolerance, stop and return
$W_{A}^{(i)}$; otherwise set $i=i+1$ and go to step 2. }
\end{enumerate}
By using a similar proof as Theorem C.1 in \cite{DM2018}, one can
show that the sequence$(W_{A}^{(i)})_{i\ge1}$ defined in Algorithm
1 is monotonically increasing and converges to $W_{A}$ uniformly.

\section*{\clearpage{}}
\begin{thebibliography}{1}
\bibitem[1]{DM2018}Dai, M., S. Kou, C. Yang, and Z. Ye. 2018. The
Overpricing of Leveraged Products: A Case Study of Dual-Purpose Funds
in China. \textit{Working Paper}. 

\bibitem[2]{Ingersoll1976}Ingersoll, J. E. 1976. A Theoretical and
Empirical Investigation of the Dual Purpose Funds: an Application
of Contingent-claims Analysis. \textit{Journal of Financial Economics}
3:83-123.

\bibitem[3]{BS}Black, F., and M. Scholes. 1973. The Pricing of Options
and Corporate Liabilities. \textit{Journal of Political Economy} 81:637-654.

\bibitem[4]{Leverage}\textit{\href{https://www.investopedia.com/terms/l/leverage.asp}{https://www.investopedia.com/terms/l/leverage.asp}}

\bibitem[5]{ETH}\textit{\href{https://github.com/ethereum/wiki/wiki/White-Paper}{https://github.com/ethereum/wiki/wiki/White-Paper}}

\bibitem[6]{USDT}\textit{\href{https://tether.to/wp-content/uploads/2016/06/TetherWhitePaper.pdf}{https://tether.to/wp-content/uploads/2016/06/TetherWhitePaper.pdf}}

\bibitem[7]{Basecoin}\textit{\href{http://www.getbasecoin.com/basecoin_whitepaper_0_99.pdf}{http://www.getbasecoin.com/basecoin\_{}whitepaper\_{}0\_{}99.pdf}}

\bibitem[8]{MAKERDAO}\textit{\href{https://makerdao.com/whitepaper/DaiDec17WP.pdf}{https://makerdao.com/whitepaper/DaiDec17WP.pdf}}
\end{thebibliography}

\end{document}
